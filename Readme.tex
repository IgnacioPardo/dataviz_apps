% Options for packages loaded elsewhere
\PassOptionsToPackage{unicode}{hyperref}
\PassOptionsToPackage{hyphens}{url}
%
\documentclass[
]{article}
\usepackage{amsmath,amssymb}
\usepackage{lmodern}
\usepackage{iftex}
\ifPDFTeX
  \usepackage[T1]{fontenc}
  \usepackage[utf8]{inputenc}
  \usepackage{textcomp} % provide euro and other symbols
\else % if luatex or xetex
  \usepackage{unicode-math}
  \defaultfontfeatures{Scale=MatchLowercase}
  \defaultfontfeatures[\rmfamily]{Ligatures=TeX,Scale=1}
\fi
% Use upquote if available, for straight quotes in verbatim environments
\IfFileExists{upquote.sty}{\usepackage{upquote}}{}
\IfFileExists{microtype.sty}{% use microtype if available
  \usepackage[]{microtype}
  \UseMicrotypeSet[protrusion]{basicmath} % disable protrusion for tt fonts
}{}
\makeatletter
\@ifundefined{KOMAClassName}{% if non-KOMA class
  \IfFileExists{parskip.sty}{%
    \usepackage{parskip}
  }{% else
    \setlength{\parindent}{0pt}
    \setlength{\parskip}{6pt plus 2pt minus 1pt}}
}{% if KOMA class
  \KOMAoptions{parskip=half}}
\makeatother
\usepackage{xcolor}
\usepackage[margin=1in]{geometry}
\usepackage{graphicx}
\makeatletter
\def\maxwidth{\ifdim\Gin@nat@width>\linewidth\linewidth\else\Gin@nat@width\fi}
\def\maxheight{\ifdim\Gin@nat@height>\textheight\textheight\else\Gin@nat@height\fi}
\makeatother
% Scale images if necessary, so that they will not overflow the page
% margins by default, and it is still possible to overwrite the defaults
% using explicit options in \includegraphics[width, height, ...]{}
\setkeys{Gin}{width=\maxwidth,height=\maxheight,keepaspectratio}
% Set default figure placement to htbp
\makeatletter
\def\fps@figure{htbp}
\makeatother
\setlength{\emergencystretch}{3em} % prevent overfull lines
\providecommand{\tightlist}{%
  \setlength{\itemsep}{0pt}\setlength{\parskip}{0pt}}
\setcounter{secnumdepth}{-\maxdimen} % remove section numbering
\ifLuaTeX
  \usepackage{selnolig}  % disable illegal ligatures
\fi
\IfFileExists{bookmark.sty}{\usepackage{bookmark}}{\usepackage{hyperref}}
\IfFileExists{xurl.sty}{\usepackage{xurl}}{} % add URL line breaks if available
\urlstyle{same} % disable monospaced font for URLs
\hypersetup{
  pdftitle={Sistema de diseño Fractal - Ignacio Pardo},
  pdfauthor={Ignacio Pardo},
  hidelinks,
  pdfcreator={LaTeX via pandoc}}

\title{Sistema de diseño Fractal - Ignacio Pardo}
\author{Ignacio Pardo}
\date{\texttt{r\ Sys.Date()}}

\begin{document}
\maketitle

\hypertarget{sistema-de-diseuxf1o-fractal---ignacio-pardo}{%
\section{Sistema de diseño Fractal - Ignacio
Pardo}\label{sistema-de-diseuxf1o-fractal---ignacio-pardo}}

\hypertarget{links}{%
\subsection{Links}\label{links}}

\href{https://github.com/IgnacioPardo/dataviz_apps}{Repo de GitHub:
https://github.com/IgnacioPardo/dataviz\_apps}
\href{https://ignaciopardo-dataviz-apps-visualizer-x6avik.streamlit.app/\#demo}{Visualización
y Demo Web:
https://ignaciopardo-dataviz-apps-visualizer-x6avik.streamlit.app/\#demo}

\hypertarget{apps-y-clasificaciuxf3n}{%
\subsection{Apps y clasificación}\label{apps-y-clasificaciuxf3n}}

Las aplicaciones que tenés en el celular son una ventana a tu
personalidad. ¿Qué dicen sobre vos? Estas son 25 aplicaciones que tengo
instaladas:

\hypertarget{comunication}{%
\paragraph{Comunication}\label{comunication}}

\begin{itemize}
\tightlist
\item
  Whatsapp
\item
  Discord \#\#\#\# Social
\item
  Instagram
\item
  Twitter
\item
  Pinterest \#\#\#\# Entertainment
\item
  Disney+
\item
  Netflix
\item
  Youtube \#\#\#\# Music
\item
  Spotify
\item
  YouTube Music
\item
  GarageBand \#\#\#\# Sports and Health
\item
  Adidas
\item
  SportClub
\item
  Fitness
\item
  Salud \#\#\#\# Identity
\item
  MercadoPago
\item
  Mi Argentina
\item
  ACAMovil
\item
  Wallet \#\#\#\# Productivity
\item
  Arc
\item
  Drive
\item
  Figma
\item
  Github
\item
  Visual Studio Code
\item
  Campus Di Tella
\end{itemize}

\hypertarget{desarrollo-de-un-pimer-sistema-de-diseuxf1o-a-mano}{%
\subsection{Desarrollo de un pimer sistema de diseño a
mano}\label{desarrollo-de-un-pimer-sistema-de-diseuxf1o-a-mano}}

\hypertarget{referencia}{%
\subsubsection{Referencia}\label{referencia}}

\hypertarget{resultado}{%
\subsubsection{Resultado}\label{resultado}}

\hypertarget{un-sistema-de-diseuxf1o-parametrizado-general}{%
\subsection{Un sistema de diseño parametrizado
general}\label{un-sistema-de-diseuxf1o-parametrizado-general}}

\hypertarget{idea-fractales}{%
\subsubsection{Idea: Fractales}\label{idea-fractales}}

\hypertarget{mandelbrot-set}{%
\paragraph{Mandelbrot Set}\label{mandelbrot-set}}

La idea de tener un sistema de diseño parametrizado es que podamos
generar distintas imágenes a partir de distintos valores.

Para ello una primera idea fue el Mandelbroth Set, un fractal que se
genera a partir de la siguiente fórmula:

\[
z_{n+1} = z_n^2 + c
\]

donde \(z_0 = 0\) y \(c\) es un número complejo.

\begin{figure}
\centering
\includegraphics{https://user-images.githubusercontent.com/65306107/238806904-a6db1db9-7a48-43de-9b05-8c8567e39f02.png}
\caption{Mandelbrot Set}
\end{figure}

Entonces podemos generar distintas imágenes a partir de distintos
valores de \(c\), y combinar otros factores, como el color de la imagen
o la resolución del fractal para aprovechar en nuestro sistema de
diseño.
\includegraphics{https://user-images.githubusercontent.com/65306107/238807160-a7888c21-6456-4db7-b140-5fe134eb3345.png}

A partir de esto plantee un sistema de diseño parametrizado que nos
permita generar distintas imágenes a partir de distintos valores de
\(C\).

\begin{figure}
\centering
\includegraphics{https://user-images.githubusercontent.com/65306107/238807263-fee40c31-2fc7-4fec-87de-842964d419b2.png}
\caption{mandelbrot\_design}
\end{figure}

\hypertarget{sistema-de-referencia-para-esta-primera-idea}{%
\paragraph{Sistema de Referencia para esta primera
idea}\label{sistema-de-referencia-para-esta-primera-idea}}

\hypertarget{el-tamauxf1o-de-la-app-afecta-el-tamauxf1o-del-fractal}{%
\subparagraph{El tamaño de la app afecta el tamaño del
fractal}\label{el-tamauxf1o-de-la-app-afecta-el-tamauxf1o-del-fractal}}

\begin{figure}
\centering
\includegraphics{https://user-images.githubusercontent.com/65306107/238807395-ef96d652-d078-4d0f-a187-24639875cda7.png}
\caption{mandelbrot\_reference\_size}
\end{figure}

\hypertarget{la-frecuencia-de-uso-de-la-app-afecta-la-cantidad-de-iteraciones-del-fractal}{%
\subparagraph{La frecuencia de uso de la app afecta la cantidad de
iteraciones del
fractal}\label{la-frecuencia-de-uso-de-la-app-afecta-la-cantidad-de-iteraciones-del-fractal}}

\begin{figure}
\centering
\includegraphics{https://user-images.githubusercontent.com/65306107/238807719-7ebadf47-45e6-4817-adac-f7a9d5358287.png}
\caption{mandelbrot\_reference\_use\_frecuency}
\end{figure}

\hypertarget{el-aprecio-por-la-app-afecta-el-color-del-fractal}{%
\subparagraph{El ``aprecio'' por la app afecta el color del
fractal}\label{el-aprecio-por-la-app-afecta-el-color-del-fractal}}

\begin{figure}
\centering
\includegraphics{https://user-images.githubusercontent.com/65306107/238807757-0196437a-857d-4077-958d-fbca8e514606.png}
\caption{mandelbrot\_reference\_likeability}
\end{figure}

\hypertarget{ampliar-el-sistema-de-diseuxf1o}{%
\subsubsection{Ampliar el sistema de
diseño}\label{ampliar-el-sistema-de-diseuxf1o}}

\hypertarget{otros-tipos-de-fractales-julia-set}{%
\paragraph{Otros tipos de fractales: Julia
Set}\label{otros-tipos-de-fractales-julia-set}}

El Mandelbrot Set es un ejemplo particular del Julia Set, por lo que
además podríamos parametrizar el tipo de fractal que queremos generar.
Un ejemplo de un fractal generado por el Julia Set a partir de
\[c = -0.1 + 0.65i\] Aprovechando entonces el sistema de diseño que ya
tenemos, podemos generar distintos fractales en función del tipo de app
que queremos representar.

Para ello elegí diferentes números complejos y los asocié a distintos
tipos de apps.\\
\textbar{} App \textbar{} Número Complejo \textbar{} \textbar{} ---
\textbar{} --- \textbar{} \textbar{} Comunication \textbar{} 0.285 +
0.01i \textbar{} \textbar{} Entertainment \textbar{} -0.8 + 0.156i
\textbar{} \textbar{} Identity \textbar{} -0.4 + 0.6i \textbar{}
\textbar{} Music \textbar{} -0.1 - 0.732i \textbar{} \textbar{}
Productivity \textbar{} -0.9 + 0i \textbar{} \textbar{} Social
\textbar{} -0.215 - 0.65i \textbar{} \textbar{} Sports and Health
\textbar{} 0.73 - 0.73i \textbar{}

De esta forma llegamos a un sistema de diseño parametrizado que nos
permite generar distintos fractales en función de distintos tipos de
apps.

\hypertarget{resultado-sistema-de-diseuxf1o-basado-en-fractales}{%
\subsubsection{Resultado: Sistema de diseño basado en
fractales}\label{resultado-sistema-de-diseuxf1o-basado-en-fractales}}

\begin{figure}
\centering
\includegraphics{https://user-images.githubusercontent.com/65306107/238807958-dd857f2d-d73d-4f62-8781-63c7cdd02894.png}
\caption{julia\_design}
\end{figure}

\hypertarget{sistema-de-diseuxf1o-final}{%
\subsubsection{Sistema de diseño
final}\label{sistema-de-diseuxf1o-final}}

\hypertarget{referencias}{%
\paragraph{Referencias}\label{referencias}}

\hypertarget{clasificaciuxf3n-de-apps}{%
\paragraph{Clasificación de apps}\label{clasificaciuxf3n-de-apps}}

El tipo de app afecta el número complejo que se usa para generar el
fractal

\begin{figure}
\centering
\includegraphics{https://user-images.githubusercontent.com/65306107/238808054-1049a269-1da3-40a1-82fc-af0ee7b55227.png}
\caption{julia\_types}
\end{figure}

\hypertarget{frecuencia-de-uso}{%
\subparagraph{Frecuencia de uso}\label{frecuencia-de-uso}}

La frecuencia de uso de la app afecta la cantidad de iteraciones del
fractal

\begin{figure}
\centering
\includegraphics{https://user-images.githubusercontent.com/65306107/238808091-e33145e5-34f5-4134-abc3-fa8b16244143.png}
\caption{julia\_use\_frecuency}
\end{figure}

\hypertarget{tamauxf1o-y-aprecio-de-la-app}{%
\subparagraph{Tamaño y aprecio de la
app}\label{tamauxf1o-y-aprecio-de-la-app}}

El tamaño y el aprecio de la app afectan el tamaño de la imagen, que se
puede interpretar como la resolución del fractal en si.

\begin{figure}
\centering
\includegraphics{https://user-images.githubusercontent.com/65306107/238808140-77ac0225-dc28-4e3e-b90c-c5e9a3b92a46.png}
\caption{julia\_size}
\end{figure}

\hypertarget{aprecio-de-la-app}{%
\subparagraph{Aprecio de la app}\label{aprecio-de-la-app}}

El aprecio de la app ademas determina el mapa de colores que se usa para
generar el fractal.

\begin{figure}
\centering
\includegraphics{https://user-images.githubusercontent.com/65306107/238808158-1ec9fb1b-a688-4d9d-a278-5e4f9aa561f1.png}
\caption{julia\_likeability}
\end{figure}

Finalmente, el sistema de diseño parametrizado desarrollado se puede
expandir para infinitos valores para generar cualquier fractal del Julia
Set para representar cualquier tipo de app.

Arme una
\href{https://ignaciopardo-dataviz-apps-visualizer-x6avik.streamlit.app/\#demo}{demo
aca}

\href{https://datavizapps.ignaciopardo.repl.co/\#demo}{Link alternativo}

\end{document}
